%% 
%% File:    PResSUv2.tex
%% Author:  Olivier Fourmaux (olivier.fourmaux@sorbonne-universite.fr)
%% 


%%%%%%%%%%%%%%%%%%%%%%%%%%%%%%%%%%%%%%%%%%%%%%%%%%%%%%%%%%%
%% Type et package

\documentclass[a4paper,12pt]{article}
\usepackage[french,english]{babel}
\usepackage{fancyhdr}
\usepackage[utf8]{inputenc}
\usepackage{cmbright}
\usepackage{epsfig}
\usepackage{calc}
\usepackage{url}
\usepackage{boxedminipage}
\usepackage{graphicx}


%%%%%%%%%%%%%%%%%%%%%%%%%%%%%%%%%%%%%%%%%%%%%%%%%%%%%%%%%%%
%%%%%%%%%%%%%%%%%%%%%%%%%%%%%%%%%%%%%%%%%%%%%%%%%%%%%%%%%%%
%% Définitions à personnaliser 

%% Pour les noms, utilisez la première lettre du prénom suivi du 
%% nom de famille (première lettre majuscule, reste en minuscule).


%%%% Indiquer le nom de l'encadrant ci-dessous:

\def\nomEncad{Prénom\_Nom\_Encadrant}

%% Si le projet est co-encadré indiquer les deux noms à la suite dans 
%% Le même champs


%%%% Indiquer les noms des étudiants participant ci-dessous:

\def\nomEtudA{Prénom\_Nom\_Etudiant1}
\def\nomEtudB{Prénom\_Nom\_Etudiant2}
\def\nomEtudC{Prénom\_Nom\_Etudiant3}
\def\nomEtudD{Prénom\_Nom\_Etudiant4}

%% Si le projet est encadré par moins de 4 étudiants laissez
%% les variables inutiles vides


%%%% Indiquer la référence (numero) et le nom du sujet ci-dessous:

\def\refProjet{33} 
\def\titreProjetCourt{Intég. XTP sous IPv6 }
\def\titreProjetLong{Intégration du protocole XTP\\ dans l'environnement IPv6}

%% Le titre court ne doit pas faire plus d'une vingtaine de caractère
%% résumez le à quelques mots essenciels


%%%% Indiquer le type de document et sa version ci-dessous:

\def\typeDoc{Rapport intermédaire/final}
 
%% - Rapport intermédaire
%% - Rapport final





%%%%%%%%%%%%%%%%%%%%%%%%%%%%%%%%%%%%%%%%%%%%%%%%%%%%%%%%%%%
%%%%%%%%%%%%%%%%%%%%%%%%%%%%%%%%%%%%%%%%%%%%%%%%%%%%%%%%%%%
%% Définitions à ne pas modifier
 
%%%%% ||| Mise en page verticale ||| 
\setlength{\voffset}{-1in} % a4:reste 297mm pour les 5 suivants:
\setlength{\topmargin}{15mm}         % avant l'en-tête
\setlength{\headheight}{20mm}        % hauteur de l'en-tête 
\setlength{\headsep}{10mm}            % entre l'en-tête et le corps
\setlength{\textheight}{220mm}       % hauteur du corps
\setlength{\footskip}{12mm}          % pied de page par rapport au corps 

%%%%% --- Mise en page horizontale ---
\setlength{\hoffset}{-1in} % a4:reste 210mm 
\setlength{\oddsidemargin}{25mm}     % entre hoffset et le corps
\setlength{\evensidemargin}{25mm}    % entre hoffset et le corps
\setlength{\marginparwidth}{0mm}     % largeur de la marge
\setlength{\marginparsep}{0mm}       % séparateur corps marge
\setlength{\textwidth}{160mm}        % largeur du corps

\def\annee{2016-17}

\renewcommand{\familydefault}{\sfdefault}


%%%%%%%%%%%%%%%%%%%%%%%%%%%%%%%%%%%%%%%%%%%%%%%%%%%%%%%%%%%
%% Début du document

\begin{document}
%\sffamily
\selectlanguage{french}



%%%%%%%%%%%%%%%%%%%%%%%%%%%%%%%%%%%%%%%%%%%%%%%%%%%%%%%%%%%
%% Définition des en-têtes et pied de pages
\pagestyle{fancyplain}
\lhead[\fancyplain{}{Master Informatique\\ UE \textbf{PRes} fév. \annee \\ \nomEncad}]
      {\fancyplain{}{Master Informatique\\ UE \textbf{PRes} \annee \\ \nomEncad}}
\chead[\fancyplain{}{\textbf{Projet \refProjet\\\titreProjetCourt}}]
      {\fancyplain{}{\textbf{Projet \refProjet\\\titreProjetCourt}}}
\rhead[\fancyplain{}{\nomEtudA\\\nomEtudB\\\nomEtudC\\\nomEtudD}]
      {\fancyplain{}{\nomEtudA\\\nomEtudB\\\nomEtudC\\\nomEtudD}}
\lfoot[\fancyplain{}{\includegraphics[width=3cm]{LOGO_SCIENCES_DEF_CMJN_med.jpg}}]
      {\fancyplain{}{\includegraphics[width=3cm]{LOGO_SCIENCES_DEF_CMJN_med.jpg}}}
\cfoot[\fancyplain{}{\textbf{\thepage/\pageref{fin}}}]
      {\fancyplain{}{\textbf{\thepage/\pageref{fin}}}}
\rfoot[\fancyplain{}{\typeDoc}]
      {\fancyplain{}{\typeDoc}}

%%%%%%%%%%%%%%%%%%%%%%%%%%%%%%%%%%%%%%%%%%%%%%%%%%%%%%%%%%%

~

      \begin{center}
        \begin{boxedminipage}{12cm}{
            \begin{center}
              ~\\\LARGE\textbf{\titreProjetLong}\\
              ~\\\large Encadrant: \textbf{\nomEncad,}\\
              ~\\\large Etudiants: \textbf{\nomEtudA, \nomEtudB, \nomEtudC, \nomEtudD}\\
              ~
            \end{center}
            }
        \end{boxedminipage}
      \end{center}

~

\tableofcontents

\newpage

\section{Cahier des charges}
Ici doit prendre place votre cahier des charges.
Pour les besoins de présentation, nous utilisons cette section comme modèle de mise en forme des rapports que vous devez fournir pour cette UE.
Dans la suite, nous allons donc plutôt développer les points suivants (qu'il faudra supprimer de vos rendus):

\subsection{Contenu des rapports}
Deux rapports sont à fournir pour l'évaluation d'un projet: l'intermédiaire et le final.


\subsubsection{Le rapport intermédiaire}
Le rapport intermédiaire se compose:
\begin{itemize}
\item d'un cahier des charges (1 page max),
\item d'un plan de développement (1 à 2 pages plus un
  diagramme de Gantt pour visualiser le séquencement des tâches réalisées par
  \textbf{chaque} étudiant),
\item d'une analyse (3 à 5 pages),
\item d'une conception \textbf{envisagée} (3 à 5 pages),
\item d'un état d'avancement du projet (1 page max justifiant l'état de
  réalisation du projet au moment du rapport).
\item d'une bibliographie réalisée grâce au \textbf{tutorat de la MIR} (1 à 2 pages),
\end{itemize}

\textbf{Le non-respect des contraintes quantitatives liées au texte du
  document sera pénalisant.}
Les graphiques (illustrations, dessins, schémas, diagrammes courbes...) sont
bienvenus pour illustrer les textes précédent. Ils viennent en plus du texte et
ne sont pas comptabilisés dans les limitations impératives indiquées
précédemment. Ils doivent rester de taille modérée pour ne pas trop alourdir
le rapport.


\subsubsection{Le rapport final}
Le rapport final se compose:
\begin{itemize}
\item du cahier des charges initial \textbf{inchangé} (celui du rapport intermédiaire),
\item du plan de développement (adapté si nécessaire, \textbf{en précisant les modifications}),
\item de l'analyse \textbf{actualisée} (3 à 5 pages),
\item de la conception \textbf{réalisée} (3 à 5 pages),
\item d'un \textbf{compte rendu du projet} (5 à 10 pages, précisant le déroulement,
  la réalisation, la validation et la livraison de ce qui avait été
  demandé),
\item de la bibliographie \textbf{actualisée} (1 à 2 pages),
\item d'annexes (si utiles pour la compréhension ou démontrant la qualité du
travail: extraits de codes commentés, manuels rédigés, etc.).
\end{itemize}



\subsection{Présentation du document}
Le document ici présenté sert de référence en terme de présentation. Des
modèles \LaTeX, OpenDocument et MSWord sont disponibles.
\textbf{Le non-respect de la charte de présentation du document sera également
pénalisant.}

Chaque rapport doit être composé d'un seul document avec les différentes
parties indiquées précédemment. Eventuellement, des annexes peuvent compléter
le document si celles-ci sont nécessaires à la bonne compréhension de
l'analyse ou du travail effectué\footnote{Attention, les annexes comportant
  du code source ne sont pertinentes que si le code proposé est largement
  commenté et relativement court. On se limitera donc à quelques échantillons
  essentiels pour la compréhension du travail réalisé.}.

La police de caractère utilisée devra être sans sérif (Helvetica, Arial, Calibri ou CM sans serif de LaTeX). La taille de celle-ci sera de \textbf{12 points}.

La première page comporte le titre (en \textbf{gras}), le nom de l'encadrant
et ceux des étudiants dans un cadre simple, centré, de 12 à 14 cm de large. Une
table des matières suit immédiatement le titre. Veillez à ce qu'il n'y ait pas
trop de niveaux visibles (la table des matières doit impérativement être limitée à la première
page).

Chaque page, y compris la première (celle du titre) comporte un en-tête, un
corps et un pied de page.

\subsubsection{En-tête d'une page}
L'en-tête indique le Master Informatique, 
le nom de l'UE avec l'année universitaire en cours puis
le nom de l'encadrant (prénom en minuscule puis nom de famille en majuscule).
Au centre de l'en-tête est indiqué en \textbf{gras} la référence du projet (son numéro) et son titre résumé (une vingtaine de caractères). 
A gauche de l'en-tête doit se trouver le nom des étudiants (initiale du prénom puis nom de famille avec seule la première lettre en majuscule pour chacun).

\subsubsection{Corps d'une page}
La zone d'écriture est de \textbf{16 cm} de large sur \textbf{22 cm} de
hauteur. Comme indiqué, la taille de la police pour le texte
normal est de 12 points, l'interlignage simple, la justification double et
les différents espacements autour des titres et paragraphes standard. Un
paragraphe commence par un retrait positif de 1 cm (sauf le premier d'une section). 

Les différentes parties seront identifiées par des titres de 1er niveau. Le
titre de deuxième et troisième niveau seront utilisés pour les structurer.
\footnote{ Attention à bien limiter la taille de votre table des
matières résultante. Il faut évidement utiliser le formatage automatique et
typer les zones de texte à sémantique particulière telles que les titres de
section pour obtenir la mise en forme standard attendue et pouvoir générer
automatiquement la table des matières.}.

\subsubsection{Pied de page}
Le pied de page comporte à droite le nouveau logo de la faculté des Sciences de Sorbonne Université sur 3 cm de large, le numéro de la page relatif par rapport au nombre total de pages en \textbf{gras} au centre et à gauche le type de rapport (intermédaire ou final).

\subsection{Transmission des rapports}
Avant tout envoi d'un rapport définitif, une version préliminaire est à envoyer à votre encadrant afin qu'il avalise votre travail. Les rapports définitifs sont à téléverser en format \textbf{PDF}\footnote{Pour vérifier si votre PDF est portable, envoyez-le en test sur une machine sous GNU/Linux et lisez-le avec le lecteur PDF local.} aux dates spécifiées et via les instructions indiquées. 

\textbf{Tout retard sera évidement pénalisant. }

\section{Plan de développement}


\section{Bibliographie}


\section{Analyse}


\section{Conception}


\section{Etat d'avancement/Compte rendu}
Selon le type de rapport (intermédaire/final).


\section{Annexe A}
Si besoin, pour le rapport final.


\section{Annexe B...}



\label{fin}

\end{document}

